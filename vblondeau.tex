%%%%%%%%%%%%%%%%%
% This is a CV created using altacv.cls (v1.1.3, 30 April 2017) written by
% LianTze Lim (liantze@gmail.com), based on the
% Cv created by BusinessInsider at http://www.businessinsider.my/a-sample-resume-for-marissa-mayer-2016-7/?r=US&IR=T
%
%% It may be distributed and/or modified under the
%% conditions of the LaTeX Project Public License, either version 1.3
%% of this license or (at your option) any later version.
%% The latest version of this license is in
%%    http://www.latex-project.org/lppl.txt
%% and version 1.3 or later is part of all distributions of LaTeX
%% version 2003/12/01 or later.
%%%%%%%%%%%%%%%%

%% If you want to use \orcid or the
%% academicons icons, add "academicons"
%% to the \documentclass options.
%% Then compile with XeLaTeX or LuaLaTeX.
% \documentclass[10pt,a4paper,academicons]{altacv}

%% Use the "normalphoto" option if you want a normal photo instead of cropped to a circle
% \documentclass[10pt,a4paper,normalphoto]{altacv}

\documentclass[10pt,a4paper]{altacv}

%% AltaCV uses the fontawesome and academicon fonts
%% and packages.
%% See texdoc.net/pkg/fontawesome and http://texdoc.net/pkg/academicons for full list of symbols.
%% When using the "academicons" option,
%% Compile with LuaLaTeX for best results. If you
%% want to use XeLaTeX, you may need to install
%% Academicons.ttf in your operating system's font %% folder.


% Change the page layout if you need to
\geometry{left=1cm,right=9cm,marginparwidth=6.8cm,marginparsep=1.2cm,top=1cm,bottom=1cm}

% Change the font if you want to.

% If using pdflatex:
\usepackage[utf8]{inputenc}
\usepackage[T1]{fontenc}
\usepackage[default]{lato}

% If using xelatex or lualatex:
% \setmainfont{Lato}

% Change the colours if you want to
\definecolor{VividPurple}{HTML}{3E0097}
\definecolor{SlateGrey}{HTML}{2E2E2E}
\definecolor{LightGrey}{HTML}{666666}
\definecolor{BaliIntense}{HTML}{5E00A4}
\definecolor{Twitter}{HTML}{1DA1F2}
\definecolor{BleuFonce}{HTML}{0A1B77}
\definecolor{BleuMoinsFonce}{HTML}{2A3B97}
\colorlet{heading}{BleuFonce}
\colorlet{accent}{BleuMoinsFonce}
\colorlet{emphasis}{SlateGrey}
\colorlet{body}{LightGrey}

% Change the bullets for itemize and rating marker
% for \cvskill if you want to
\renewcommand{\itemmarker}{{\small\textbullet}}
\renewcommand{\ratingmarker}{\faCircle}

%% sample.bib contains your publications
\addbibresource{blondeau.bib}

\newcommand*\leftright[3]{%
	\leavevmode
	\rlap{#2}%
	\hspace{#1}%
	#3}


\begin{document}
\name{Vincent Blondeau}
\tagline{PhD In Software Quality}
% Cropped to square from https://en.wikipedia.org/wiki/Marissa_Mayer#/media/File:Marissa_Mayer_May_2014_(cropped).jpg, CC-BY 2.0
%\photo{2.5cm}{mmayer-wikipedia-cc-by-2_0}
\personalinfo{%
  % Not all of these are required!
  % You can add your own with \printinfo{symbol}{detail}
  \email{v.blondeau@bbox.fr}
  \location{Ile-de-France, France}
  \location{San Francisco, CA, USA}
%  \homepage{marissamayr.tumblr.com/}
%  \twitter{@marissamayer}
  \linkedin{linkedin.com/in/vincent-blondeau}
   \github{github.com/VincentBlondeau} % I'm just making this up though.
%   \orcid{orcid.org/0000-0000-0000-0000} % Obviously making this up too. If you want to use this field (and also other academicons symbols), add "academicons" option to \documentclass{altacv}
Driving License
}

%% Make the header extend all the way to the right, if you want.
\begin{fullwidth}
\makecvheader
\end{fullwidth}

%% Provide the file name containing the sidebar contents as an optional parameter to \cvsection.
%% You can always just use \marginpar{...} if you do
%% not need to align the top of the contents to any
%% \cvsection title in the "main" bar.


\cvsection[vblondeau-p1sidebar]{Who I Am}

Ph.D. in computer science, I am looking for a job in innovation and software quality.

\cvsection{Experience}

\cvevent{Industrial Postdoc}{Lam Research Corporation}{Jan. 2018 -- Dec. 2018}{Fremont, California, USA}
\begin{itemize}	
	\item Industrial Postdoc.:\\ \textbf{Improvement of usability of the software}
	\item Develop tool to create executables from Pharo applications
	\item Implement a MQTT backend
	\item Design a Sequence Diagram Generator
\end{itemize}

\divider

\cvevent{Ph.D. Student / Engineer design and development}{Inria / Worldline}{Oct. 2014 -- Sep. 2017}{Seclin, France}
\begin{itemize}	
	\item Industrial Ph.D.:\\ \textbf{Test Selection Habits of Developers in a Large IT Company}
	\item Develop plugins in Java to automatically select tests after a change in the source code
	\item Implement Pharo Server Backend for test selection
	\item Extend Moose static model
\end{itemize}


\divider

\cvevent{Qualitative Analysis of a System (Internship)}{Worldline}{Mar. 2014 --- Aug. 2014 }{Seclin, France}
\begin{itemize}
	\item Integrated in a transversal team of the company
	\item Explore tracks to improve the maintenance and the software quality in collaboration with the Inria RMod team
	\item Develop tools to validate software architecture thanks to Moose (based on Pharo)
\end{itemize}

\divider

\cvevent{A REST API for Moose (Internship)}{Synectique}{May 2013 ---  Aug. 2013}{Villeneuve d'Ascq, France}
\begin{itemize}
	\item Synectique is a startup originated of Inria specialized in maintenance and source code analysis \item Develop a REST server to access a source code analysis model generated in the Moose environment
	\item Create a web client with Amber (a Smalltalk for the web) showing the features of the server
\end{itemize}

\divider

\cvevent{Experiments on the Android Kernel (Internship)}{Laboratoire d’Informatique Fondamentale, Inria}{Jul. 2012}{Lille, France}
\begin{itemize}
	\item Study of multi-core scheduling algorithm
\end{itemize}


%\divider
%
%\cvevent{Internship period}{IT Department, TF1 Headquarters}{Jun. 2010 – Aug. 2010}{Boulogne-Billancourt, France}
%\begin{itemize}
%	\item  Management of the computer equipment 
%	\item  Computer’s preparation, setup, and delivery 
%	\item  Software update 
%\end{itemize}

%\divider

%\cvevent{Internship period}{Research \& Innovation Department, Bouygues Bâtiment International}{Jul 2009 --- Aug 2009}{Challenger, Saint Quentin en Yvelines, France}
%\begin{itemize}
%	\item Database update (Bouygues’s Electronic Document Management System)  
%	\item 3D modeling on Sketch Up and layout in Google Earth  
%	\item Digitalization and recording in the EDMS of technical documents 
%\end{itemize}

%
%\cvsection{A Day of My Life}
%
%% Adapted from @Jake's answer from http://tex.stackexchange.com/a/82729/226
%% \wheelchart{outer radius}{inner radius}{
%% comma-separated list of value/text width/color/detail}
%\wheelchart{1.5cm}{0.5cm}{%
%  10/13em/accent!30/Sleeping \& dreaming about work,
%  25/9em/accent!60/Public resolving issues with Yahoo!\ investors,
%  5/12em/accent!10/New York \& San Francisco Ballet Jawbone board member,
%  20/12em/accent!40/Spending time with family,
%  5/8em/accent!20/Business development for Yahoo!\ after the Verizon acquisition,
%  30/9em/accent/Showing Yahoo!\ employees that their work has meaning,
%  5/8em/accent!20/Baking cupcakes
%}

\clearpage

\cvsection[page2sidebar]{Publications}



\nocite{*}



\printbibliography[heading=pubtype,title={\printinfo{\faFileTextO}{Ph.D. Thesis}},type=thesis]

\divider

\printbibliography[heading=pubtype,title={\printinfo{\faFilesO}{Journal Articles}}, type=article]

\divider

\printbibliography[heading=pubtype,title={\printinfo{\faGroup}{Conference Proceedings}},type=inproceedings]


\cvsection{Teaching}

\cvevent{Databases: Modelisation \& SQL - Master 1}{Polytech Lille}{2017}{Villeneuve d'Ascq, France}

\divider

\cvevent{Databases: Modelisation \& SQL - Master 1}{Polytech Lille}{2016}{Villeneuve d'Ascq, France}

\cvsection{Research Reviews}

 {\large\color{emphasis}ICPC 2017 --- External reviewer\par}
 25th IEEE International Conference on Program Comprehension

\divider

 {\large\color{emphasis}ICSOFT-PT 2015 --- External reviewer\par}
10th International Conference on Software Paradigm Trends



%% If the NEXT page doesn't start with a \cvsection but you'd
%% still like to add a sidebar, then use this command on THIS
%% page to add it. The optional argument lets you pull up the
%% sidebar a bit so that it looks aligned with the top of the
%% main column.
% \addnextpagesidebar[-1ex]{page3sidebar}


\end{document}
